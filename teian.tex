\chapter{提案手法}
\thispagestyle{fancy}

\section{使用装置}
\subsection{Kinect for Windows}
Microsoftから販売された,コントローラーを用いずに身体の動き,ジェスチャー,音声などによって
操作を可能にする周辺機器である.

Kinectには,赤外線センサー,8ビット3チャンネル(RGB)の画像データを取得するRGBカメラ,
Kinectからの距離(深度)の画像データを取得する深度画像センサー,音の発生場所を求める
音源位置推定が可能な音声マイクが搭載されている.
また,Kinectの最も特徴的な機能が「姿勢認識技術」である,人間の全身を認識してその動きによる操作をしている.
これにより,深度画像をもとに,人体のパーツがどこにあるのかを推測することができる\cite{kinect}.

本研究では,Kinect for Windows v1を使用した.

\begin{figure}[b]
    \centering
    \includegraphics[width=9cm]{image/kinect.png}
    \caption{Kinect for Windows\cite{kinect}}
  \label{kinect}
\end{figure}

\clearpage

\subsection{プロジェクター}
投影を行う際に使用した.


\section{開発環境}

\begin{itemize}
    \item OS: Windows 10
    \item 統合開発環境: Visual Studio 2017
    \item プログラミング言語: C++
    \item ライブラリ: OpenNI2, NiTE2, OpenCV, OpenGL
\end{itemize}

\section{システムの概要}
本研究では,スポーツに焦点を当て野球とサッカーの動作を行えるプロジェクションマッピングを実装した.


\subsection{表示画面}
以下の9つの画面を出力する.

\begin{itemize}
    \item Ball: スポーツで用いるボール.図\ref{ball}.
    \item Color: RGBカメラの映像.図\ref{color}.
    \item Depth: 深度カメラの映像.図\ref{depth}.
    \item User: 人領域の映像.図\ref{user}.
    \item Combination: 投影用.図\ref{combination}.
    \item Combination\_PC: PC用.図\ref{combination}.
    \item Skeleton: 人の骨格情報.図\ref{skeleton}.
    \item Gray: Color画面にグレースケールを実行した画面.図\ref{gray}.
    \item Cascade: 黒い背景画像にGray画面を合成した画面.図\ref{cascade}.
\end{itemize}

\clearpage

\begin{figure}[p]
    \begin{minipage}{0.5\hsize}
     \begin{center}
      \includegraphics[width=4cm,height=6cm]{image/ball.png}
     \end{center}
     \caption{Ballの画面表示}
     \label{ball}
    \end{minipage}
    \begin{minipage}{0.5\hsize}
     \begin{center}
      \includegraphics[width=4cm,height=6cm]{image/color.png}
     \end{center}
     \caption{Colorの画面表示}
     \label{color}
    \end{minipage}
\end{figure}

\begin{figure}[p]
    \begin{minipage}{0.5\hsize}
     \begin{center}
      \includegraphics[width=4cm,height=6cm]{image/Depth.png}
     \end{center}
     \caption{Depthの画面表示}
     \label{depth}
    \end{minipage}
    \begin{minipage}{0.5\hsize}
     \begin{center}
      \includegraphics[width=4cm,height=6cm]{image/User.png}
     \end{center}
     \caption{Userの画面表示}
     \label{user}
    \end{minipage}
\end{figure}

\clearpage

\begin{figure}[p]
    \begin{minipage}{0.5\hsize}
     \begin{center}
      \includegraphics[width=4cm,height=6cm]{image/Combination.png}
     \end{center}
     \caption{CombinationとCombination\_PCの画面表示}
     \label{combination}
    \end{minipage}
    \begin{minipage}{0.5\hsize}
     \begin{center}
      \includegraphics[width=4cm,height=6cm]{image/Skeleton.png}
     \end{center}
     \caption{Skeletonの画面表示}
     \label{skeleton}
    \end{minipage}
\end{figure}

\begin{figure}[p]
    \begin{minipage}{0.5\hsize}
     \begin{center}
      \includegraphics[width=4cm,height=6cm]{image/gray.png}
     \end{center}
     \caption{Grayの画面表示}
     \label{gray}
    \end{minipage}
    \begin{minipage}{0.5\hsize}
     \begin{center}
      \includegraphics[width=4cm,height=6cm]{image/cascade.png}
     \end{center}
     \caption{Cascadeの画面表示}
     \label{cascade}
    \end{minipage}
\end{figure}

\clearpage

\subsection{スケルトンナンバー}
Kinect v1において,スケルトンは図\ref{num}に示すように番号が割り当てられている.

\begin{figure}[htbp]
    \centering
    \includegraphics[height=7cm]{image/Skeleton_num.png}
    \caption{スケルトンナンバー}
  \label{num}
\end{figure}

\begin{table}[h]
    \centering
    \begin{tabular}{|lc|lc|} \hline
      0 & 頭 & 8 & 胴体 \\ 
      1 & 首 & 9 & 左腰 \\
      2 & 左肩 & 10 & 右腰 \\
      3 & 右肩 & 11 & 左膝 \\
      4 & 左肘 & 12 & 右膝 \\
      5 & 右肘 & 13 & 左足 \\
      6 & 左手 & 14 & 右足 \\
      7 & 右手 &  &  \\ \hline
    \end{tabular}
    \caption{スケルトンナンバーと体の位置関係}
    \label{num}
\end{table}



ユーザにCombination画面の投影を行い,ユーザのスケルトン座標に応じてインタラクティブに
プロジェクションマッピングを変化させる.
システム起動時のスケルトンの認識の不安定さを防ぐために,ユーザは「気を付け」のポーズをとる事にする.
このポーズは以下に示す状態のポーズである.(図\ref{num}参照).

\begin{itemize}
    \item 左肩と右肩の$y$座標の差が$100mm$未満
    \item 左肘と右肘の$y$座標の差が$150mm$未満
\end{itemize}

モードの一連の動作が終了すると,初期化を行う.
それによって,連続して投影を行うことが可能にした.
それぞれのモードに切り替わる前は,画面は黒いままである.

\subsection{野球モード}
ピッチャーの一連の動作を行う際に,野球ボールと野球場が投影される.
ボールは投げる際に音がなり,飛距離が伸びるにつれ,徐々に小さくなる.
ボールの位置が$x$座標のしきい値(被験者から見て画面左端)より大きくなった場合,ボールは消えて初期化される.

以下にその手順を示す.

\begin{enumerate}
    \item 胸の付近で両手を構える(以下の6つの条件を満たす必要がある)
        \begin{itemize}
            \item 左肘の$x$座標と胴の中心の$x$座標の差が$180mm$未満
            \item 左肘の$y$座標と胴の中心の$y$座標の差が$180mm$未満
            \item 右肘の$x$座標と胴の中心の$x$座標の差が$180mm$未満
            \item 右肘の$y$座標と胴の中心の$y$座標の差が$180mm$未満
            \item 首の$y$座標と左手の$y$座標の差が$180mm$未満
            \item 首の$y$座標と右手の$y$座標の差が$180mm$未満
        \end{itemize}
    \item 頭より上に来るように右手を振り上げるとボールが出現し動き出す(以下の条件で認識する).
        \begin{itemize}
            \item 振り上げた右手の$y$座標が頭の中心の$y$座標よりも高い
        \end{itemize}
\end{enumerate}



\begin{figure}[h]
    \centering
    \includegraphics[width=6cm]{image/baseball.png}
    \caption{野球モードを実行した際のCombinationの画面表示}
  \label{baseball}
\end{figure}


\clearpage

\subsection{サッカーモード}
リフティングの一連の動作を行うと,サッカーボールとサッカー場が投影される.
ボールを蹴り上げる際には音がなる.また,
ボールの位置が$y$座標のしきい値(画面下)より低くなった場合,ボールは消え初期化される.

以下にその手順を示す.

\begin{enumerate}
    \item 右膝を右腰ほどの高さになるように蹴り上げるとボールが出現する(以下の条件で認識する).
        \begin{itemize}
            \item 右膝の$y$座標が右腰から$300mm$低い位置より高く蹴り上げると,ボールが出現し動き出す.            
        \end{itemize}
    \item 右膝を下ろすとボールは下がり続ける.
\end{enumerate}


\begin{figure}[h]
    \centering
    \includegraphics[width=6cm]{image/soccer.png}
    \caption{サッカーモードを実行した際のCombinationの画面表示}
  \label{baseball}
\end{figure}

\vspace{1cm}

また,リフティングの動きは,以下の物理演算を使用している.

$y$座標は上向きを正とし,プロジェクションマッピングの画面中央を$y=0$とする.

\clearpage

\begin{figure}[htbp]
    \centering
    \includegraphics[width=9cm]{image/butsuri2.png}
    \caption{リフティングのボールの動き}
  \label{butsuri}
\end{figure}



\[
    \begin{cases}
        v_0,v_0': 初速度 & \\
        t: 時間 & \\
        g: 重力加速度(g=9.8)とする. &
    \end{cases}
\]

\vspace{1cm}

\begin{itemize}
    \item[a] ボールの位置が$y \geqq 0$の場合
        \begin{equation}
            y=v_0t-\frac{1}{2}gt^2
        \end{equation}
        \begin{equation}
            v_0=25とすると,y=25*t-\frac{1}{2}gt^2
        \end{equation}
    \item[b,c] ボールの位置が$y < 0$の場合
        \begin{itemize}
            \item[b] 一定速度で落下
            \item[c] ボールを蹴り上げる動作をした時
                \begin{equation}
                    y=v_0't-\frac{1}{2}gt^2
                \end{equation}
        \end{itemize}
\end{itemize}

\vspace{1.5cm}

\begin{figure}[htbp]
    \centering
    \includegraphics[width=6cm]{image/ballenzan.png}
    \caption{ボールを蹴り上げた時点でのボールの位置}
  \label{enzan}
\end{figure}

\vspace{1cm}


\begin{itemize}
    \item ボールの最高点の位置をy=h(図\ref{butsuri}参照)
    \item ボールを蹴り上げた時点でのボールの位置をy=-h(図\ref{butsuri},図\ref{enzan}参照)と仮定
\end{itemize}


連続したリフティング動作を行えるようにするため,aでのボールの最高点に,
再び蹴り上げたボールの高さをそろえる.これを実現させるため,以下の計算を行う(図\ref{butsuri}参照).


\begin{itemize}
    \item[aより]
        \begin{equation}
            v^2-v_0^2=2(-g)h
        \end{equation}

        最高点の速さは$v=0$より 

        \begin{equation}
            0-v_2^2=2(-g)h
        \end{equation}

        よって

        \begin{equation}
            v_0=\sqrt{2gh}
        \end{equation}

    \item[cより]
        \begin{equation}
            v^2-v_0'^2=2(-g)(2h)
        \end{equation}

        最高点の速さは$v=0$より 

        \begin{equation}
            0-v_0'^2=2(-g)(2h)
        \end{equation}
        よって

        \begin{equation}
            v_0'=\sqrt{4gh}
        \end{equation}

    \item[以上より]
        \begin{equation}
            v_0'=\sqrt{2}v_0
        \end{equation}

        よって$v_0'$は$v_0$の$\sqrt{2}$倍である. 

        \begin{equation}
            v_0'=25*\sqrt{2} \fallingdotseq 35.4 
        \end{equation}

        より

        \begin{equation}
            y=35.4*t-\frac{1}{2}gt^2
        \end{equation}

    \item[dより] ボールが閾値(画面下)より低くなった場合ボールは消える. 

\end{itemize}

\vspace{1.5cm}

\subsection{カスケード分類器を用いたユーザのポーズ認識}
Kincetには,スケルトントラッキングの対象となる人物を認識した
人物から無作為に選択する問題が存在している\cite{hitogomi}.
骨格座標を用いた手法では,ユーザがKinectの視野から隠れた後,再度Kinectの視野内に
入った場合,ユーザの骨格座標の再追従が上手く行われない時があった.
その問題を解決する手段として,Gray画面にcascade分類器を用いてユーザの検出を行う手法を提案する.


\clearpage

\subsubsection{AdaBoost}
ブースティングは,弱い分類器を順次生成し,
それらを組み合わせて強い分類器を作成する学習アルゴリズムである\cite{boosting}.

本研究では,さまざまなブースティング手法の中でAdaBoostと呼ばれる手法を使用した.
AdaBoostは,学習プロセス中に分類器の認識率に適応的に重み付けすることにより学習することにより,
高精度で分類器を作成する手法である\cite{adaboost}.

図\ref{adaboost}において,$h_T(x)$は$T$個目の特徴量を指し,$\alpha$は重みを指す.

\vspace{1.2cm}


\begin{figure}[h]
    \centering
    \includegraphics[width=12cm]{image/adaboost.png}
    \caption{AdaBoost\cite{adafig}}
  \label{adaboost}
\end{figure}

\vspace{1.2cm}

\begin{itemize}
    \item ブースティングのおおまかな流れ\cite{suugaku}
        \begin{enumerate}
            \item $弱学習器h_1(x)を作る$
            \item $h_1(x)の結果を考慮して,次の弱学習器h_2(x)を作る$
            \item $以下順々に,h_{t-1}(x)の結果を考慮して,次の弱学習器h_tを作る$
            \item $h_Tまで作ったらh_1(x)からh_T(x)までをまとめて強学習器h(x)を作る$
        \end{enumerate}
        \clearpage
    
    \item AdaBoostの重み
        \begin{itemize}
            \item 各学習器の重み$\alpha_t$の決め方($E_t$: 学習器$h_t$の誤差)
                \begin{equation}
                    \alpha_t=\frac{1}{2}\ln(\frac{1-E_t}{E_t})
                \end{equation}
                

                $h_t$の誤分類が大きくなればなるほど,$\alpha_t$が小さくなり$h_t$の重要度は小さくなる.

            \item $t$回目の学習におけるサンプル$i$の重み$w_i^{(t)}$の決め方
                \begin{equation}
                    w_i^{(t)}=C_tw_i^{(t-1)}e^{-y_i\alpha_th_t(x_i)}
                \end{equation}
            
        \end{itemize}
\end{itemize}


\subsubsection{画像の特徴抽出}
カスケード分類器は,学習時に特徴量の抽出を行う. 
以下に3つの特徴量の抽出方法を示す.

\begin{itemize}
    \item Haar-like特徴量
    \item Local Binary Pattern (LBP)特徴量
    \item Histogram of Oriented Gradients (HOG)特徴量
\end{itemize}

本研究では,LBP特徴量を用いた.


\clearpage
\subsubsection{LBP特徴量}
LBPとは,画像の認識と分類に使用できる特徴量の1つである\cite{adafig}\cite{lbp}.
LBPは回転変動などには弱いが,照明変化の影響を受けにくく,また高速計算が可能といった特徴がある.

LBPの計算は$3*3$ピクセルサイズの画素領域に着目し行う.
まず中心の輝度値と周辺8近傍の画素の輝度値を比較する.その8近傍の内,
輝度値が中心の輝度値以上のとき1,それ以外を0とする.ここにマスクを掛け合わせ,
その総和を求めこの値が中心画素の輝度値と置き換える. 
マスクとは,左上から時計回りに$2^n$の重みを割り振ったものである (図\ref{lbpfig}中,右から2番目の図).
この作業を全画素に対して行い,できた画像をLBP画像と呼ぶ.
こうして求めたLBP特徴量を用いて物体を認識する.

\vspace{1.5cm}

\begin{figure}[h]
    \centering
    \includegraphics[width=13cm]{image/lbpfig.png}
    \caption{LBPの計算方法\cite{adafig}}
  \label{lbpfig}
\end{figure}

%\begin{equation}
%    LBP=1+16+32+64+128=241
%\end{equation}

\clearpage
\subsubsection{サッカーモードにおけるカスケード分類器の作成}
サッカーモードにおいて,ユーザが膝でボールを蹴り上げている動作を行っている
画像をポジテイブ画像,
ユーザがそれ以外の動作を行っている画像をネガテイブ画像として学習を行いカスケード分類器を作成した.

\vspace{1.5cm}

\begin{figure}[h]
    \begin{minipage}{0.5\hsize}
     \begin{center}
      \includegraphics[width=3cm,height=4cm]{image/posi.png}
     \end{center}
     \caption{ポジティブ画像}
     \label{posi}
    \end{minipage}
    \begin{minipage}{0.5\hsize}
     \begin{center}
      \includegraphics[width=3cm,height=4cm]{image/nega2.png}
     \end{center}
     \caption{ネガティブ画像}
     \label{nega}
    \end{minipage}
\end{figure}







