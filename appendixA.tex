\chapter{プログラム}
\thispagestyle{fancy}

本研究で用いたプログラム

\vspace{0.5cm}

main.cpp
\begin{lstlisting}

#pragma comment(linker, "/SUBSYSTEM:WINDOWS /ENTRY:mainCRTStartup")


#include "stdafx.h"

#include<stdio.h>
#include <OpenNI.h> // OpenNI2 Header file
#include <NiTE.h> // NITE2 Header file
#include <opencv2/opencv.hpp>
#include <opencv/highgui.h>

#include <GL/gl.h>
#include <GL/glu.h>
#include <GL/glut.h>

#include <string.h>

#include <mmsystem.h>
#pragma comment(lib,"winmm.lib") // "winmm.libというライブラリをリンかでリンクする

#define RANGE 200


// 画面のサイズ
#define WIDTH 640 
#define HEIGHT 480

// 物理演算
#define g 9.8
#define v0 


#if 1

using namespace cv;
using namespace std;

// ballの平行移動用
float x = 0.0f;
float trans_x = 0.0f; // ボールのx軸の動き
float trans_y = 0.0f; // ボールのy軸の動き


float scale_x = 0.2; // ボールの大きさ
float scale_y = 0.2;
float scale_z = 0.2;

bool flag = false;


// ボール
GLfloat green[] = { 0.0, 1.0, 0.0, 1.0 }; // ballcolor 緑
GLfloat lightpos[] = { 200.0, 150.0, -500.0, 1.0 }; // ライトの位置

bool baseballFlag = false; // ボールを出さない
bool soccerFlag = false;
bool soccerFlag_up = false;
bool baseballFlag_re = false; // restart
bool soccerFlag_re = false;

// 物理演算
double F, a;// 力と加速度
double y = 0;// y座標
double t = 0;// 時間

// 背景画像
cv::Mat img, img2;




void printString(float x, float y, char* str, int length) {
	float z = -1.0f;
	glRasterPos3f(x, y, z);

	for (int i = 0; i < length; i++) {
		glutBitmapCharacter(GLUT_BITMAP_HELVETICA_18, str[i]);
	}
}

// 文字列描画
static void DrawString(String str, int w, int h, int x0, int y0)
{
	glColor3d(0, 250, 0); // 画面上にテキスト描画
	glRasterPos2f(x0, y0); // 文字列の位置(左上原点のスクリーン座標系, 文字列の左下がこの位置になる)

	int size = (int)str.size(); // str:文字列
	for (int i = 0; i < size; ++i) {
		char ic = str[i];
		glutBitmapCharacter(GLUT_BITMAP_9_BY_15, ic);
	}
}

// OpenNI2とNITE2を使ってKinectのColor,Depth,User,Skeleton,Combination,Combination_PC,Ballを表示する
void display(void)
{


	// 背景画像を読み込む
	img = cv::imread("baseball_back.jpg", 1);
	img2 = cv::imread("soccer_back.jpg", 1);

	// エラー処理
	if (img.data == NULL) {
		printf("img read error!\n");
	}

	cv::setUseOptimized(true);

	// Initialize OpenNI2
	openni::OpenNI::initialize();

	// Initialize NITE2 
	nite::NiTE::initialize();



	// Device Open
	openni::Status statusOpenNI = openni::STATUS_OK;
	openni::Device device;
	statusOpenNI = device.open(openni::ANY_DEVICE);
	if (statusOpenNI != openni::STATUS_OK) {
		cerr << "Error : openni::Device::open" << endl;
		return;
	}

	// Color Stream Create and Open
	openni::VideoStream colorStream;
	colorStream.create(device, openni::SENSOR_COLOR);
	statusOpenNI = colorStream.start();
	if (statusOpenNI != openni::STATUS_OK) {
		cerr << "Error : openni::VideoStream::start( COLOR )" << endl;
		return;
	}


	// Depth Stream Create and Open
	openni::VideoStream depthStream;
	depthStream.create(device, openni::SENSOR_DEPTH);
	statusOpenNI = depthStream.start();
	if (statusOpenNI != openni::STATUS_OK) {
		cerr << "Error : openni::VideoStream::start( DEPTH )" << endl;
		return;
	}

	// User Tracker Create and Open
	nite::Status statusNITE = nite::STATUS_OK;
	nite::UserTracker userTracker;
	statusNITE = userTracker.create(&device);
	if (statusNITE != nite::STATUS_OK) {
		cerr << "Error : nite::UserTracker::create" << endl;
		return;
	}

	// Setting enable Synchronize
	device.setDepthColorSyncEnabled(true);

	// Set Registration Mode Depth to Color
	// But Kinect doesn't support this Function!
	device.setImageRegistrationMode(openni::IMAGE_REGISTRATION_DEPTH_TO_COLOR);


	cv::Mat grayMat;
	cv::Mat colorMat;
	cv::Mat depthMat;
	cv::Mat userMat;
	cv::Mat skeletonMat;


	// カスケード分類器読み込み
	CascadeClassifier cascade;
	// ここに読み込むカスケード分類器のパスを記述
	cascade.load("");
	// ユーザのポーズ情報を格納場所
	vector<Rect> faces;



	// Windowのサイズ
	cv::namedWindow("Color", cv::WINDOW_NORMAL);
	cv::namedWindow("Gray", cv::WINDOW_NORMAL);
	cv::namedWindow("Depth", cv::WINDOW_NORMAL);
	cv::namedWindow("User", cv::WINDOW_NORMAL);
	cv::namedWindow("Skeleton", cv::WINDOW_NORMAL);

	// Table of Colors
	cv::Vec3b color[7];
	color[0] = cv::Vec3b(255, 255, 255);
	color[1] = cv::Vec3b(255, 0, 0);
	color[2] = cv::Vec3b(0, 255, 0);
	color[3] = cv::Vec3b(0, 0, 255);
	color[4] = cv::Vec3b(255, 255, 0);
	color[5] = cv::Vec3b(255, 0, 255);
	color[6] = cv::Vec3b(0, 255, 255);

	float tmp_x[15];
	float tmp_y[15];
	float tmp_z[15];

	int baseball_status = -1;
	int soccer_status = -1;

	while (1) {
		// Retrieve Frame from Color Stream (8bit 3channel)
		openni::VideoFrameRef colorFrame;
		// Retrieve a Frame from Stream
		colorStream.readFrame(&colorFrame);
		// cv::cvtColor(入力画像, 出力画像, 色空間を変換する値);
		if (colorFrame.isValid()) {
			// Retrieve a Data from Frame 
			colorMat = cv::Mat(colorStream.getVideoMode().getResolutionY(), colorStream.getVideoMode().getResolutionX(), CV_8UC3, reinterpret_cast<uchar*>(const_cast<void*>(colorFrame.getData()))); 
			// Change the order of the pixel RGB to BGR 
			cv::cvtColor(colorMat, colorMat, CV_RGB2BGR); 
			// Change the order of the pixel RGB to GRAY
			cv::cvtColor(colorMat, grayMat, CV_BGR2GRAY); 

		}


		// Retrieve Frame from Depth Stream (16bit 1channel)
		openni::VideoFrameRef depthFrame;
		// Retrieve a Frame from Stream
		depthStream.readFrame(&depthFrame); 
		if (depthFrame.isValid()) {
			// Retrieve a Data from Frame
			depthMat = cv::Mat(depthStream.getVideoMode().getResolutionY(), depthStream.getVideoMode().getResolutionX(), CV_16UC1, reinterpret_cast<ushort*>(const_cast<void*>(depthFrame.getData()))); 
			depthMat.convertTo(depthMat, CV_8UC1, -255.0f / 10000, 255.0);
			// Convert the pixel 0~10000 to 0~255
		}

		// Retrieve User Frame from UserTracker
		nite::UserTrackerFrameRef userFrame;
		// Retrive a Frame form Tracker
		userTracker.readFrame(&userFrame); 
		const nite::UserId* pUserId = userFrame.getUserMap().getPixels();
		// Retrive UserId from Frame
		int width = userFrame.getUserMap().getWidth();
		int height = userFrame.getUserMap().getHeight();
		// 画像を格納する
		userMat = cv::Mat(height, width, CV_8UC3, cv::Scalar(255, 255, 255));

		if (userFrame.isValid()) {
			for (int y = 0; y < height; y++) {
				for (int x = 0; x < width; x++) {
					// 色を付ける
					userMat.at<cv::Vec3b>(y, x) = color[*pUserId]; 
					pUserId++;
				}
			}
		}

		// Retrieve Skeleton Frame from UserTracker
		skeletonMat = cv::Mat(height, width, CV_8UC3, cv::Scalar(255, 255, 255));
		// Retrieve User from User Frame
		const nite::Array<nite::UserData>& users = userFrame.getUsers(); 
		for (int count = 0; count < users.getSize(); count++) {
			// Start Skeleton Tracking a new User
			if (users[count].isNew()) {
				userTracker.startSkeletonTracking(users[count].getId());
			}
			// Retrieve Skeleton from Tracking User ( who is Not Lost and Visible User )
			else if (!users[count].isLost() && users[count].isVisible()) {
				// Retrieve Skeleton form User
				const nite::Skeleton& skeleton = users[count].getSkeleton(); 
				if (skeleton.getState() == nite::SkeletonState::SKELETON_TRACKED) {
					for (int position = 0; position < 15; position++) {
						const nite::SkeletonJoint& joint = skeleton.getJoint((nite::JointType)position);
						// Retrieve Joint from Skeleton ( Total 14 joint )
						const nite::Point3f& point = joint.getPosition();
						// Retrieve three-dimensional position of the Joint
						cv::Point2f registPoint;
						// Registration Joint Position to Depth
						userTracker.convertJointCoordinatesToDepth(point.x, point.y, point.z, &registPoint.x, &registPoint.y);
						cv::circle(skeletonMat, registPoint, 10, static_cast<cv::Scalar>(color[count + 1]), -1, CV_AA);

						tmp_x[position] = point.x;
						tmp_y[position] = point.y;
						tmp_z[position] = point.z;


						// 動きの判別を行う
						// 初期化
						if (position == 8 && (baseball_status == -1 || soccer_status == -1)) {
							// 「気を付け」の姿勢
							if ((abs(tmp_y[2] - tmp_y[3]) < 100) && ((abs(tmp_y[4] - tmp_y[5]) < 120))) {
								trans_x = 0;
								trans_y = 0;
								baseball_status = 0;
								soccer_status = 0;
								printf("Junbi\n");
							}
						}

						printf("xxyyyyyy:%d %d %d\n", position, (int)point.x, (int)point.y);

						// 野球(右手)
						// RANGE200 1=首 4=左ひじ 5=右ひじ 6=左手 7=右手 8=胴体の中心
						if (position == 8 && baseball_status == 0 && soccerFlag == false) {
                            // 野球モード
                            // 両手がお腹あたりにくる
							if (((abs(tmp_x[4] - tmp_x[8]) < RANGE) && (abs(tmp_y[4] - tmp_y[8]) < RANGE) && ((abs(tmp_x[5] - tmp_x[8]) < RANGE) && (abs(tmp_y[5] - tmp_y[8]) < RANGE)) && ((abs(tmp_y[1] - tmp_y[6]) < RANGE) && ((abs(tmp_y[1] - tmp_y[7]) < RANGE))))) {  
								baseball_status = 1;
								printf("Kamae\n");
							}
						}

						// 野球(右手)
						if (position == 7 && baseball_status == 1 && soccer_status == 0) {
							// 右手が頭より上にくる
							if (point.y > tmp_y[0]) {
								if (position == 7 && tmp_y[0] < tmp_y[7]) {
									baseballFlag = true;
									baseball_status = 2;
								}
								printf("Tewoageru\n");
							}
						}

						// 野球(左手)
						if (position == 6 && baseball_status == 1 && soccer_status == 0) {
							// 左手が頭より上にくる
							if (point.y > tmp_y[0]) {
								if (position == 6 && tmp_y[0] < tmp_y[6]) {
									baseballFlag = true;
									baseball_status = 2;
								}
								printf("Tewoageru\n");
							}
						}

						// サッカー(右脚)
						if (position == 12 && soccer_status == 0 && baseballFlag == false) {
							// 右膝が右腰より上 <-ココを分類器を用いて認識する
							if (tmp_y[12] > tmp_y[10] - 300) {
								soccerFlag = true;
								soccer_status = 1;
								printf("Hizawoageru\n");
							}
						}

						// サッカー(左脚)
						if (position == 11 && soccer_status == 0 && baseballFlag == false) {
							// 左膝が左腰より上 <-ココを分類器を用いて認識する
							if (tmp_y[11] > tmp_y[9] - 300) {
								soccerFlag = true;
								soccer_status = 1;
								printf("Hizawoageru\n");
							}
						}

						printf("(10)%f,  (11)%f,  (13)%f,  (14)%f\n", tmp_y[10], tmp_y[11], tmp_y[13], tmp_y[14]);
						printf("(9)%f,  (12)%f,\n", tmp_y[9], tmp_y[12]);

						// 最初に戻る
						if (trans_x > 100 || trans_y < -50) {
							baseballFlag = false;
							soccerFlag = false;
							baseball_status = 0;
							soccer_status = 0;
							trans_x = 0;
							trans_y = 0;

							// 元のボールの大きさを戻す
							scale_x = 0.2; 
							scale_y = 0.2;
							scale_z = 0.2;

							// 初期化
							t = 0; 
							soccerFlag_up = false;
							printf("Modoru\n");
						}
					}
				}
			}
		}


		// ボールの描画
		glClear(GL_COLOR_BUFFER_BIT | GL_DEPTH_BUFFER_BIT);
		glViewport(0, 0, WIDTH, HEIGHT);
		glMatrixMode(GL_PROJECTION);
		glLoadIdentity();


		// 視野角,アスペクト比(ウィンドウの幅/高さ),描画する範囲(最も近い距離,最も遠い距離)
		gluPerspective(30.0, (double)WIDTH / (double)HEIGHT, 1.0, 1000.0);
		glMatrixMode(GL_MODELVIEW);
		glLoadIdentity();

		// 視点の設定
		//カメラの座標
		gluLookAt(150.0, 100.0, -200.0, 
			// 注視点の座標
			0.0, 0.0, 0.0, 
			// 画面の上方向を指すベクトル
			0.0, 1.0, 0.0);
		
		// ライトの設定
		glLightfv(GL_LIGHT0, GL_POSITION, lightpos);

		// マテリアルの設定
		glMaterialfv(GL_FRONT, GL_DIFFUSE, green);

		// ボールの表示
		if (baseballFlag || soccerFlag) {
			// 野球
			if (baseballFlag == true && soccerFlag == false) {

				// ボールが出てくる位置
				glTranslatef(trans_x - 30, trans_y + 55, 0.0f);
				// ボールの拡大縮小 glScalef(0.2, 0.15, 0.2)
				glScalef(scale_x, scale_y - 0.05, scale_z); 
				glutSolidSphere(40.0, 16, 16);

				// ボール縮小
				scale_x = scale_x * 0.88; 
				scale_y = scale_y * 0.88;
				scale_z = scale_z * 0.88;

				// ボールがx軸に沿って動く
				trans_x += 12.f; 

				// サウンド
				PlaySound(L"baseball.wav", NULL, SND_FILENAME | SND_SYNC | SND_ASYNC); 
				printf("Baseball\n");

			}

			// サッカー
			if (soccerFlag == true && baseballFlag == false) {
				// ボールが出てくる位置
				glTranslatef(trans_x - 35, trans_y, 0.0f);
				// ボールの拡大
				glScalef(0.4, 0.3, 0.4); 
				glutSolidSphere(40.0, 16, 16);

				// 2回目以降(振り出しに戻す)
				if (soccerFlag_up == true && trans_y > 0) {
					// 初期化
					t = 0;
					soccerFlag_up = false;
				}

				// ボールが膝以上の高さ
				if (trans_y >= 0) { 
					trans_y = 25 * t - 0.5 * g * t * t;
				}

				// ボールが膝より下の高さ
				else if (trans_y < 0) {
					// ボールが落ちた時
					if (soccer_status == 1) {
						soccerFlag_up = true;
						// 初期化
						t = 0;
					}

					// 脚を上げる
					if (soccerFlag_up == true) {
						// 35.4 = 25 * 1.414
						trans_y = 35.4 * t - 0.5 * g * t * t;
						//ボールが上がる
						trans_y += 5.0;
						// サウンド
						PlaySound(L"soccer.wav", NULL, SND_FILENAME | SND_SYNC | SND_ASYNC);
					}
					else {
						// ボールが落ち続ける
						trans_y = -5 * 1.2 * t;
					}

					printf("Soccer\n");
				}
				t += 0.8;
				soccer_status = 0;
			}

			int bits = glutGet(GLUT_WINDOW_BUFFER_SIZE);

		}

		// 背景の切り替え
		GLint oldMatrixMode;

		glGetIntegerv(GL_MATRIX_MODE, &oldMatrixMode);
		glMatrixMode(GL_PROJECTION);
		glPushMatrix();
		glLoadIdentity();
		glRasterPos2i(-1, -1);
		glPixelZoom(1, 1);

		glDrawPixels(img.cols, img.rows, GL_RGB, GL_UNSIGNED_BYTE, img.data);
		glPopMatrix();
		glMatrixMode(oldMatrixMode);
		glClear(GL_DEPTH_BUFFER_BIT);

		char str[20];

		sprintf(str, "%d", baseball_status);


		/*
		//status表示
		glPushMatrix(); 
		printString(80, 65, str, strlen(str));
		glPopMatrix();
		*/

		// OpenGLで描画された画面をOpenCVのcv::Mat形式の画像として取得する
		// Ball (Combination)
		// 3チャンネルのデータ
		int type = CV_8UC3;
		int format = GL_BGR_EXT;

		glReadBuffer(GL_FRONT);
		cv::Mat out_img(cv::Size(WIDTH, HEIGHT), type);
		glReadPixels(0, 0, WIDTH, HEIGHT, format, GL_UNSIGNED_BYTE, out_img.data);
		cv::flip(out_img, out_img, 0);
		
		// Matの要素を取得
		for (int i = 0; i < 3 * WIDTH; i++) {
			for (int j = 0; j < HEIGHT; j++) {
				if (userMat.at<uchar>(j, i) != 0 && out_img.at<uchar>(j, i) == 0) {
					if (baseball_status >= 1) {
						out_img.at<uchar>(j, i) = img.at<uchar>(j, i);
					}
					else if (soccerFlag == true) {
						out_img.at<uchar>(j, i) = img2.at<uchar>(j, i);
					}
				}
				else if (out_img.at<uchar>(j, i) == 0) {
					out_img.at<uchar>(j, i) = userMat.at<uchar>(j, i);
					
				}
			}
		}

		// 画像を90度回転
		cv::transpose(out_img, out_img);

		// 画面のサイズの拡大縮小
		resize(out_img, out_img, Size(), 3, 1.5); 
		glutSwapBuffers();


		// 分類器の為のグレースケール背景
		cv::Mat bg = cv::imread("haikei.png");
		cv::Mat bg2;

		// 張り付ける位置
		int x = 300, y = 150;
		
		// 背景からはみ出しいないかチェック
		CV_Assert((x >= 0) && (x + colorMat.cols <= bg.cols));
		CV_Assert((y >= 0) && (y + colorMat.rows <= bg.rows));
		
		// colorMatのサイズ変更
		cv::resize(colorMat, colorMat, cv::Size(), 0.2, 0.25);
		
		// 前面画像の画素を背景にコピー
		cv::Mat roi = bg(cv::Rect(x, y, colorMat.cols, colorMat.rows));
		colorMat.copyTo(roi);
		
		// colorMatをグレースケール化
		cv::cvtColor(bg, bg2, CV_BGR2GRAY);


		// カスケードファイルに基づいて人のポーズを検知する.検知した情報をベクトルfacesに格納
		cascade.detectMultiScale(bg2, faces, 1.1, 1, 0, Size(50, 50));

		// "faces.size()"分ループを行う
		for (int i = 0; i<faces.size(); i++)
		{
			// 検出したポーズを赤色矩形で囲む
			rectangle(bg2, Point(faces[i].x, faces[i].y), Point(faces[i].x + faces[i].width, faces[i].y + faces[i].height), Scalar(0, 0, 255), 3, CV_AA); 
			/*
			//サッカー
			//soccerFlag = true;
			//soccer_status = 1;
			//printf("Hizawoageru\n");
			*/
		}
		
		// 検出した個数が1、つまり動作の検出に成功したとき
		/*if(face.size() > 0){ボールを表示する}*/


		// 画面表示 
		//imshow(ウィンドウ名, 表示される画像);

		// Color画面表示
		cv::imshow("Color", colorMat);
		
		// Gray画面表示
		cv::imshow("Gray", grayMat);
		
		// Depth画面表示
		cv::imshow("Depth", depthMat);
		
		// User画面表示
		cv::imshow("User", userMat);
		
		// Skeleton画面表示
		cv::imshow("Skeleton", skeletonMat);
		
		// Combination画面表示
		cv::imshow("Combination", out_img);
		
		// Combination_PC画面表示
		cv::imshow("Combination_PC", out_img);
		
		// Cascade画面表示
		cv::imshow("cascade", bg2);
		
		// Kensyutsu画面表示
		cv::imshow("kensyutu", bg2);
											   
                                               
        
		// Press the Escape key to Exit
		if (cv::waitKey(30) == VK_ESCAPE) {
			break;
		}
	}

	// Shutdown Application
	cv::destroyAllWindows();
	colorStream.stop();
	depthStream.stop();
	colorStream.destroy();
	depthStream.destroy();
	userTracker.destroy();
	device.close();
	openni::OpenNI::shutdown();
	nite::NiTE::shutdown();
}

void idle(void)
{
	if (flag) { x -= 0.5f; }
	else { x += 0.5f; }
	if (x > 50.0f)flag = true;
	if (x < -50.0f)flag = false;
	Sleep(1);
	glutPostRedisplay();
}

void Init() {
	glClearColor(0.3f, 0.3f, 0.3f, 1.0f);
	glEnable(GL_DEPTH_TEST);
	glEnable(GL_LIGHTING);
	glEnable(GL_LIGHT0);
}

int main(int argc, char *argv[])
{
	glutInitWindowPosition(100, 100);
	glutInitWindowSize(WIDTH, HEIGHT);
	glutInit(&argc, argv);
	glutInitDisplayMode(GLUT_RGBA | GLUT_DOUBLE);
	glutCreateWindow("Ball");
	glutDisplayFunc(display);

	glutIdleFunc(idle);
	Init();
	glClearColor(0, 0, 0, 0);
	glutMainLoop();

	return 0;
}

#endif
\end{lstlisting}



\vspace{1cm}



stdafx.h
\begin{lstlisting}
// stdafx.h : 標準のシステム インクルード ファイルのインクルード ファイル、または
// 参照回数が多く、かつあまり変更されない、プロジェクト専用のインクルードファイルを記述。

#pragma once

#include "targetver.h"
#include <stdio.h>
#include <tchar.h>
\end{lstlisting}



\vspace{1.5cm}



targetver.h
\begin{lstlisting}
#pragma once

// SDKDDKVer.h をインクルードすると、利用できる最も上位の Windows プラットフォームが定義されます。

// 以前の Windows プラットフォーム用にアプリケーションをビルドする場合は、WinSDKVer.h をインクルードし、
// SDKDDKVer.h をインクルードする前に、サポート対象とするプラットフォームを示すように _WIN32_WINNT マクロを設定します。

#include <SDKDDKVer.h>
\end{lstlisting}


\vspace{1.5cm}

stdafx.cpp
\begin{lstlisting}
// stdafx.cpp : 標準インクルード Sample.pch のみを
// 含むソース ファイルは、プリコンパイル済みヘッダーになります。
// stdafx.obj にはプリコンパイル済み型情報が含まれます。

#include "stdafx.h"
\end{lstlisting}

