\chapter{評価実験}
\thispagestyle{fancy}


\section{評価実験}
サッカーモードにおけるカスケード分類器ごとの認識率などの評価にする?


サッカーモードに使用する学習データを以下に示す.

maxFalseAlarmRateが0.5の場合,これは現在のモデルでは,トレーニングプロセス
中に1000個のうち5個の正のサンプルが誤って分類される事を許可する.
また,minHitRateは各ステージでのヒット率である.

\vspace{1cm}

\begin{table}[h]
  \centering
  \scalebox{1.0}[1.0]{
   \begin{tabular}{cc}	\hline
     ポジティブ画像 & 1200  \\    
     ネガティブ画像 & 345  \\
     ステージ数 & 11  \\ \hline
   \end{tabular}
  } 
  \caption{サッカーモードにおける学習データ1}
  \label{soccerstudy}
\end{table}


\begin{table}[h]
  \centering
  \scalebox{1.0}[1.0]{
   \begin{tabular}{cc}	\hline
     ポジティブ画像 & 1200  \\    
     ネガティブ画像 & 345  \\
     maxFalseAlarmRate & 0.4 \\
     ステージ数 & 14  \\ \hline
   \end{tabular}
  } 
  \caption{サッカーモードにおける学習データ2}
  \label{soccerstudy}
\end{table}


\begin{table}[h]
  \centering
  \scalebox{1.0}[1.0]{
   \begin{tabular}{cc}	\hline
     ポジティブ画像 & 1200  \\    
     ネガティブ画像 & 345  \\
     minHitRate & 0.998 \\
     ステージ数 & 14  \\ \hline
   \end{tabular}
  } 
  \caption{サッカーモードにおける学習データ3}
  \label{soccerstudy}
\end{table}


\begin{table}[h]
  \centering
  \scalebox{1.0}[1.0]{
   \begin{tabular}{cc}	\hline
    ポジティブ画像 & 1200  \\    
    ネガティブ画像 & 345  \\
    maxFalseAlarmRate & 0.4 \\
    minHitRate & 0.998 \\
    ステージ数 & 16  \\ \hline
   \end{tabular}
  } 
  \caption{サッカーモードにおける学習データ4}
  \label{soccerstudy}
\end{table}


\begin{table}[h]
  \centering
  \scalebox{1.0}[1.0]{
   \begin{tabular}{cc}	\hline
    ポジティブ画像 & 1200  \\    
    ネガティブ画像 & 345  \\
    maxFalseAlarmRate & 0.35 \\
    ステージ数 & 14  \\ \hline
   \end{tabular}
  } 
  \caption{サッカーモードにおける学習データ5}
  \label{soccerstudy}
\end{table}


\begin{table}[h]
  \centering
  \scalebox{1.0}[1.0]{
   \begin{tabular}{cc}	\hline
    ポジティブ画像 & 1200  \\    
    ネガティブ画像 & 345  \\
    maxFalseAlarmRate & 0.35 \\
    minHitRate & 0.998 \\
    ステージ数 & 16  \\ \hline
   \end{tabular}
  } 
  \caption{サッカーモードにおける学習データ6}
  \label{soccerstudy}
\end{table}
