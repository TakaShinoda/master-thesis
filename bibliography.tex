\begin{thebibliography}{99}
\thispagestyle{fancy}
%%% 概要 %%%
\bibitem{hoge} hoge

%%% はじめに %%%
\bibitem{tppm} 小笠航, 片寄晴弘. ``TPPM(Take Part in Projection Mapping):タブレット端末を用いた多人数参加型 プロジェクションマッピングアプリケーション'', 2014年

%%% 関連研究 %%%
\bibitem{once} ``ディズニーランド「ワンス」11月で終了へ'' シネマトゥデイ. [Online] https://www.cinematoday.jp/news/N0090846.
\bibitem{olympic} ``オリンピック大会を支えた製品・技術'', パナソニック. [Online]  https://www.panasonic.com/global/olympic/ja/rio/solution/dlp.html.
\bibitem{amuro} ``自己最長ツアー「LIVE STYLE 2016-2017」に見た、安室奈美恵というパフォーマーの矜持'', SPICE. [Online] https://spice.eplus.jp/articles/116764.
\bibitem{kirameku} ``きらめく日本技術'', niponica. [Online] http://web-japan.org/niponica/niponica14/ja/feature/feature03.html.
\bibitem{rioprojection} ``リオの会場に咲いたパナのプロジェクションマッピング'' オリパラスポーツイノベーション. [Online] https://style.nikkei.com/article/DGXMZO06971870X00C16A9UP2000/.
\bibitem{iryou} ``プロジェクションマッピング技術を手術ナビゲーションシステムに応用'', 国立研究開発法人 日本医療研究開発気候. [Online] https://www.amed.go.jp/pr/2017\_seikasyu\_02-16.html.


%%% 提案手法 %%%
\bibitem{kinect} 西林孝, 小野憲史, ``キネクトハッカーズマニュアル''. ラトルズ. 2011年.
\bibitem{hitogomi} 濱砂雅人, 伊藤暢浩, 幸塚義之, ``人ごみにおけるKinectセンサの誤認追跡の改善について'', 30th Fuzzy System Symposium, Kochi, September 1-3, 2014.
\bibitem{boosting} Nobuyuki Nishiuchi, Takuma Suzuki, Kimihiro Yamanaka ``Development of Work Posture Evaluation System Using Boosting'' J Jpn Ind Manage Assoc 62,51-58,2011.
\bibitem{adaboost} Yoav Freund, Robert E Schapire, ``A Decision-Theoretic Generalization of On-Line Learningand an Application to Boosting'', journal of computer and system sciences 55, 119139 (1997).
\bibitem{adafig} 森まどか, 會澤要, 鈴木拓央, 小林邦和 ``カスケード型分類器を用いた照明環境変化にロバストなサッカーボール認識'', 人工知能学会研究会資料. JSAI Technical Report SIG-Challenge-051-5 (5/4).
\bibitem{lbp} Ojala T, Pietikainen M, Harwood D, ``Performance evaluation of texture measures with classification based on Kullback discrimination of distributions.'', Proc. 12th International Conference on Pattern Recognition (ICPR 1994), Jerusalem, Israel. Vol I, 582-585. (1994).



\end{thebibliography}    