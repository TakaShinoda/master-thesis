\chapter*{概要}

近年,エンターテインメントコンピューティングが日本でますます注目されている.
その中でも本研究では,プロジェクションマッピングに着目した.

プロジェクションマッピングは,プロジェクターを使用して空間と映像を合成することにより、
新しい空間を作成する技術である。
多くの人がダンサーのパフォーマンスとプロジェクションマッピングが融合した作品に魅了されたことがあるだろう.

ただし、これらの作業では、パフォーマーがプロジェクションマッピングの画像オブジェクトの座標に
正確に動きを合わせる必要があるため多くの人にとっては困難であると考える.

そこで,プロジェクションマッピングを見る人々だけでなく,パフォーマーも楽しませるために,
ユーザの動きに応じてインタラクティブに変化する参加型のプロジェクションマッピングを提案する.

本研究では,スポーツに焦点を当てKinectを用いて
ユーザの骨格座標に応じて野球のピッチングとサッカーのリフティング動作を行える
プロジェクションマッピングを実装した.

また,ユーザがKinectの視野から隠れた後,再度Kinectの視野内に
入った場合,ユーザの骨格座標の再追従が上手く行われない問題に対しては,
カスケード分類器を用いてユーザの検出を行う手法でアプローチした.

本研究で提案したプロジェクションマッピングを5名のユーザに体験してもらい,
評価を得るためにアンケートを実施した.

アンケート結果より,
プロジェクションマッピングを見る人々だけでなく,
ユーザも楽しめるかについては一定の評価を得る事ができた.

今後の課題としては,操作を分かりやすくするためにチュートリアル画面の実装や,
体の小さな子供や体の不自由な方にも体験してもらえるようにすることなどが挙げられる.