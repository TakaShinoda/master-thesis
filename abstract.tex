\chapter*{概要}

近年,エンターテインメントコンピューティングが日本でますます注目されている.
その中でも本研究では,プロジェクションマッピングに着目した.
プロジェクションマッピングは,プロジェクターを使用して空間と映像を合成することにより、
新しい空間を作成する技術である。
多くの人がダンサーのパフォーマンスとプロジェクションマッピングが融合した作品に魅了されたことがあるだろう.
ただし、これらの作業では、パフォーマーがプロジェクションマッピングの画像オブジェクトの座標に
正確に動きを合わせる必要があるため多くの人にとって困難であると考える.

そこで,プロジェクションマッピングを見る人々だけでなく,パフォーマーも楽しませるために,
パフォーマーの動きに応じてインタラクティブに変化する参加型のプロジェクションマッピングを提案する.

本研究では,スポーツに焦点を当てKinectを用いて
野球のピッチングとサッカーのリフティング動作を行えるプロジェクションマッピングを実装した.
また,ユーザがKinectの視野から隠れた後,再度Kinectの視野内に
入った場合,ユーザの骨格座標の再追従が上手く行われない時がある問題に対しては
カスケード分類器を用いてユーザの検出を行う手法でアプローチした.

本研究で提案したプロジェクションマッピングに対する評価を得るために
X人の被験者に対してアンケーの実施を行った.

評価結果をかく

今後の課題としては,

