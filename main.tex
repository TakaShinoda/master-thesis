\documentclass[uplatex, onecolumn,12pt]{jsbook}


\usepackage[dvipdfmx]{graphicx}
\usepackage{latexsym}
\usepackage{bmpsize}
\usepackage{url}
\usepackage{comment}

\def\Underline{\setbox0\hbox\bgroup\let\\\endUnderline}
\def\endUnderline{\vphantom{y}\egroup\smash{\underline{\box0}}\\}

\newcommand{\ttt}[1]{\texttt{#1}}

\title{
{\Large 令和元年度 修士論文} \\[4cm]
\LARGE KinectとOpenCVを用いた
     \\参加型プロジェクションマッピングの試作 \\[4cm]
}
\author{宮崎大学 大学院 工学研究科修士課程\\
工学専攻 機械・情報系コース\\
篠田 貴大\\
\\
指導教員 坂本 眞人 准教授
}
\date{令和2年1月27日}

\begin{document}
\maketitle

\clearpage


\section{はじめに}
\begin{figure}[t]
    \begin{center}
        \includegraphics[width=7cm]{image/syokuji_computer.png}
        \caption{パソコンの前でご飯を食べる人のイラスト}
        \label{fig:syokuji_computer}
    \end{center}
\end{figure}

パソコンの前でご飯を食べることはよくある。パソコンの前でご飯を食べる人のイラストを図\ref{fig:syokuji_computer}に示す。
このイラストは、規約の範囲内であれば、個人、法人、商用、非商用問わず無料で利用できることでおなじみの、{\bf かわいいフリー素材 いらすとや}\cite{irasutoya}より引用した。

\section{おわりに}
やっぱり{\bf いらすとや}のイラストはすばらしい。

\begin{thebibliography}{99}
    \bibitem{irasutoya} いらすとや, last access 2019.6.13 \url{https://www.irasutoya.com/}



\end{thebibliography}

\end{document}